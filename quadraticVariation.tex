\documentclass[12pt]{article}

%%%%%% Packages
\usepackage{amsmath, amssymb}


%%%%%% New commands
% iid text above the distribution sign
\newcommand{\iid}{\overset{\mathrm{\mathrm{iid}}}{\sim}}
% Expectation symbol
\DeclareMathOperator*{\E}{\mathbb{E}}
% Partial differentiation shorthand
\newcommand{\dpart}[2]{\frac{\partial #1}{\partial #2}}


%%%%%% Opening details
\title{Quadratic Variation}
\author{Treacher}


\begin{document}

\date{}
\maketitle

\begin{abstract}

\noindent Just some notes about quadratic variation, summarised from\\ https://benjaminwhiteside.com/2017/01/26/quadratic-variation/

\end{abstract}

\section{Definition}

Quadratic variation is defined as the sum of squared changes in a process $X_t$:

\begin{equation} \label{eq:QuadraticVariation}
	[X]_t =\lim\limits_{\mathrm{max}(\Delta t)\to 0}\,\sum_{i=1}^n\left(X_{t_i} - X_{t_{i-1}}\right)^2
\end{equation}

\noindent where $\Delta t = t_i - t_{i-1}$ for any $t_i$ which is the $i^\mathrm{th}$ partition of the full time space $t\in [0,T]$. The limit is over the maximum time increment going to zero (implying the different increments can tend to zero at different rates?)



\section{Using a Wiener process}

Properties of the Wiener process:
\begin{itemize}
	\item $W_0 = N(0,0)$
	\item $W_t - W_s = N(0, t-s)$
	\item Var$(W_t - W_s) = t-s$
	\item $W_t, W_{t+1}, ... W_{t+n} \iid$
\end{itemize}

\noindent We can therefore write the variance of a Wiener process in terms of the expectation value knowing that Var$(x) = \E[(x - \E[x])^2]$, and the expectation of a Wiener process is always 0 to say:

\begin{equation} \label{eq:VartoE}
	\mathrm{Var}(W_t - W_s) = \E[(\{W_t - W_s\} - \E[W_t - W_s])^2] = \E[(W_t - W_s)^2] = t - s
\end{equation}

\noindent Now consider the sum of squared differences in a Wiener process denoted by $Q_p$

\begin{equation}
	Q_p=\sum_{i=1}^n\left(W_{t_{i+1}} - W_{t_i}\right)^2
\end{equation}

\noindent and take the expected value to recover something similar to the variance expression:

\begin{equation}
	\E[Q_p] = \E\left[\sum_{i=1}^n\left(W_{t_{i+1}} - W_{t_i}\right)^2\right] =\sum_{i=1}^n t_{i+1} - t_i
\end{equation}

\noindent having changed the order of the expectation and summation, and observing the expression at the end of equation \ref{eq:VartoE}.\\
\\
The sum of the independent time increments is by definition equal to the to the final value of the time, such that

\begin{equation}
	\E[Q_p] = t.
\end{equation}

\noindent From this we conclude that ``Brownian motion accumulates quadratic variation at a rate of 1 per unit of time."

\section{Ito integral}


The definition of an Ito integral of a stochastic process $\xi(t)$ is (do separate summary of where this comes from at some point):

\begin{equation} \label{eq:itoIntegral}
	\int_{0}^{t} \xi(t)\mathrm{d}W_s = \lim\limits_{h\to 0}\sum_{k=1}^{i_t}\xi(t_{k-1})\cdot \left(W_{t_k}-W_{t_{k-1}}\right)
\end{equation}

\noindent If we take the stochastic process $\xi(t)=1$ and square both sides then the right hand side of the above is just $\E[Q_p]$ which we know is equal to $t$. This allows us to say

\begin{equation}
	\int_{0}^{t}\mathrm{d}W_s^2 = t
\end{equation}

\noindent or by differentiating,

\begin{equation}
	\mathrm{d}W_s^2 = \mathrm{d}t.
\end{equation}


\section{Application to $dt^2$}

Consider the quadratic variation as above but applied to a time process $t$ instead of a stochastic process $X$:

\begin{equation}
	[t]_t =\lim\limits_{n\to \infty}\,\sum_{i=1}^n\left(t_{i+1} - t_i\right)^2
\end{equation}

\noindent We know there are $n$ time increments, which must have equal length of $t/n$ which allows the above to be written as


\begin{equation}
	[t]_t =\lim\limits_{n\to \infty}\,\sum_{i=1}^n\left(\frac{t}{n}\right)^2=\lim\limits_{n\to \infty}\left(\frac{t}{n}\right)^2\,\sum_{i=1}^n=\lim\limits_{n\to \infty}\frac{t^2}{n}=0
\end{equation}

\noindent which reveals the quadratic variation of time $[t]_t=0$. Using the Ito integral in equation \ref{eq:itoIntegral} with respect to $t$, we can then say


\begin{equation}
	\int_{0}^{t}\mathrm{d}s\cdot\mathrm{d}s =[t]_t=0
\end{equation}

\noindent or equivalently via differentiation:

\begin{eqnarray}
	\mathrm{d}t^2=0
\end{eqnarray}

\end{document}
